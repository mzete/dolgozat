\chapter{Bevezető}
A szakdolgozat témája robotok egyensúlyozását megvalósító rendszerek tanulmányozása. Ezen rendszereket nevezhetjük zárt vagy nyílt ciklusosnak. Nyílt ciklusos rendszer esetén nincs visszacsatolás, ezért nem megfelelő olyan problémák megoldására, amelyeknél ellenőriznünk kell a rendszer kimenetének eredményét, tehát nem szükséges a visszacsatolás. Egyszerű problémák megoldása esetén használatos. A zárt ciklusos rendszerek visszacsatolással működnek, irányítható a rendszer. Visszacsatolásos működés(\ref{fig:closeLoop} ábra) annyit tesz, hogy a kimenet hatására egy visszajelzést kapunk, amely tulajdonképpen a rendszer bemenete lesz, így a visszajelzés szabályozza a kimenetet és ezáltal komplex feladatok megvalósítására alkalmas.

\begin{figure}[!htb]
	\centering
	\pgfimage[width=0.9\linewidth]{images/close_loop_controller}
	\caption[Zárt ciklusos rendszer működési elve.]
	{Zárt ciklusos rendszer működési elve \href{https://en.wikipedia.org/wiki/Control\_theory}{https://en.wikipedia.org/wiki/Control\_theory}}
	\label{fig:closeLoop}
\end{figure}

Zárt ciklusos rendszerek esetén szabályzó algoritmusról beszélünk, amelyek használata számos technológiában jelen van. Ilyenek közé sorolható a \texttt{Swagway}\footnote{\href{https://swagway.com}{https://swagway.com}} illetve a hasonló szerkezetű és működésű \texttt{Segway}\footnote{\href{http://www.segwaymagyarorszag.com/}{http://www.segwaymagyarorszag.com/}} .

A dolgozat során olyan projekt kerül bemutatásra, amely a LEGO MINDSTORMS EV3\cite{mindstormsEv3} készletből épített kétkerekű egyensúlyozó robot irányítását teszi lehetővé hálózaton keresztül, telefonos alkalmazás segítségével. 

A kétkerekű robot a Gyro Boy\footnote{\href{http://robotsquare.com/wp-content/uploads/2013/10/45544\_gyroboy.pdf}{http://robotsquare.com/wp-content/uploads/2013/10/45544\_gyroboy.pdf}} modell alapján készült el. Főbb komponensei közé sorolható az EV3 vezérlőegység\footnote{\href{http://lego.wikia.com/wiki/45500\_EV3\_Intelligent\_Brick}{http://lego.wikia.com/wiki/45500\_EV3\_Intelligent\_Brick}}, giroszkóp szenzor\footnote{\href{http://shop.lego.com/en-US/EV3-Gyro-Sensor-45505}{http://shop.lego.com/en-US/EV3-Gyro-Sensor-45505}} és a két nagy szervomotor\footnote{\href{http://lego.wikia.com/wiki/45502\_EV3\_Large\_Servo\_Motor}{http://lego.wikia.com/wiki/45502\_EV3\_Large\_Servo\_Motor}}, amelyek egy tengelyen helyezkednek el. Az előbb említett elemekről bővebben szó lesz a \ref{sec:ROBOT:motorok} illetve \ref{sec:ROBOT:szenzorok} fejezetben. Mivel a kerekek egy tengelyen helyezkednek el ezért instabil a szerkezete és könnyen, rövid idő alatt elveszti egyensúlyi állapotát. E probléma megoldására a robot dőlési szöge, szög változásának a sebessége és a robot sebessége 0-hoz kell, hogy tartson, valamint annak érdekében, hogy egy helyben próbálja megtartani egyensúlyát a robot pozíciója is 0-hoz kell közelítsen. 

A probléma megoldására alkalmas a PID\footnote{\href{https://en.wikipedia.org/wiki/PID\_controller}{https://en.wikipedia.org/wiki/PID\_controller}} szabályzó algoritmus használata, amely ipari körökben elterjedt. Viszonylag egyszerű a felépítése, kezelhetősége és az implementálhatósága. A PID szabályzó zárt ciklusos rendszer, melynek a bemeneti értéke a hiba, amely az elvárt érték és az aktuális érték különbsége. Esetünkben mivel 0-t kell közelítsünk annak érdekében, hogy ne veszítse el egyensúlyi állapotát a robot, ezért az elvárt érték örökké 0 lesz, vagyis a hibát négy komponens alkotja: a robot dőlési szöge, szögsebessége, a robot sebessége és pozíciója, amelyek külön-külön súlyozva vannak. A szög és szögsebesség érték meghatározásához a giroszkóp szenzort használjuk és a szervomotorok beépített szenzorjai által lekérhető fordulat szám segítségével számoljuk ki a robot sebességét és pozícióját.

A robot irányításának érdekében szükséges a PID szabályzó algoritmus módosítása és esetleges újabb szabályzók bevezetése annak érdekében, hogy irányítás alatt ne veszítse el az egyensúlyi állapotát. A felhasználónak lehetőséget ad a projekt részeként elkészített Android alkalmazás, hogy hálózaton, socketeken keresztül csatlakozzon a robothoz és az irányításnak megfelelő adatokat továbbítsa. Ezen adatok beviteli módját egy "touch joystick" teszi lehetővé, amellyel négy irányba lehetséges a robot vezérlése. Az adatok védelmét, a tovább bővíthetőséget, illetve a szerializációt a Google Protocol Buffers\footnote{\href {https://developers.google.com/protocol-buffers/}{https://developers.google.com/protocol-buffers/}} biztosítja. A Protocol Buffers platform és nyelvfüggetlen, könnyen kezelhető és gyors. Lehetőséget nyújt az adatok tetszőleges felépítésére, amelynek a forráskódját egy speciális generátor segítségével könnyen kigenerálható. E strukturált adatok írását illetve olvasását biztosítja a generált kód.

Az alkalmazás és a robot közti kapcsolat létrehozásának automatizálására Cling-UPnP(Universal Plug and Play)~\cite{upnp} könyvtárat használjuk, amely SSDP(Simple Service Discovery Protocol)\footnote{\href{https://en.wikipedia.org/wiki/Simple\_Service\_Discovery\_Protocol}{https://en.wikipedia.org/wiki/Simple\_Service\_Discovery\_Protocol}} protokollt használ.

A robot szabályzó algoritmusa Java-ban íródott, amelynek a futtatási környezetét a leJOS firmware biztosítja. Linux alapú  és magába foglalja a JVM-t (Java virtual machine), amely lehetővé teszi, hogy a robot programozható legyen Java-ban. Számos firmware-t fejlesztettek ki annak érdekében, hogy a LEGO MINDSTORMS által fejlesztett vezérlőegységek programozhatóak legyenek magas szinten is. Pár ismert firmware: leJOS\footnote{\href{http://www.lejos.org}{http://www.lejos.org}} José Solórzano hozta létre 1999 végén, nyílt forráskódú, támogatja az objektum orientált programozást, ROBOTC~\cite{robotc} támogatja a C programozást, ev3dev~\cite{ev3dev}, amely a szkript nyelveket támogatja (Phyton, NodeJS, Ruby). 

Az előbb említett firmware-k teszik lehetővé a komplex feladatok megoldását, amelyek nem lehetségesek az EV3 alapértelmezett rendszerével. E rendszerhez a LEGO MINDSTORMS biztosított egy grafikus felületet, amely a kisebb korosztály számára készült, hogy "programozhassák" a saját kezűleg épített robotokat.

A dolgozat négy fejezetből áll. Az első fejezet által betekintést nyerünk a dolgozat témájába. 

A második fejezet röviden bemutatja a LEGO megalakulását, a LEGO MINDSTORMS által kifejlesztett generációkat, majd bemutatja az EV3 készlethez tartozó motorokat és szenzorokat. Tartalmazza azon eszközök részletes leírását, amelyek a projekt során felhasználásra kerültek.

A harmadik fejezet által bemutatásra kerül a PID szabályzó működése és használata e projekt esetén.

A negyedik fejezet célja, hogy bemutassa e szakdolgozat alatt létrehozott projekt működését és az elkészített Android mobil alkalmazást. Valamint a projekt elkészítése során felmerült problémákat, ezek megoldását, illetve lehetséges megoldását.