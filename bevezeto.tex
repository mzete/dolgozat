\chapter{Bevezető}
A szakdolgozat által bemutatásra kerül egy projekt, amely a LEGO MINDSTORMS EV3\cite{mindstormsEv3} készletből épített kétkerekű egyensúlyozó robot irányítását teszi lehetővé hálózaton keresztül, telefonos alkalmazás segítségével. Emellet sor kerül a robot főbb elemeinek a bemutatása, felhasznált technológiák és a vezérlést kezelő Android alkalmazás.

A kétkerekű robot a Gyro Boy\footnote{\href{http://robotsquare.com/wp-content/uploads/2013/10/45544\_gyroboy.pdf}{http://robotsquare.com/wp-content/uploads/2013/10/45544\_gyroboy.pdf}} modell alapján készült el. Főbb komponensei közé sorolható az EV3 vezérlőegység\footnote{\href{http://lego.wikia.com/wiki/45500\_EV3\_Intelligent\_Brick}{http://lego.wikia.com/wiki/45500\_EV3\_Intelligent\_Brick}}, giroszkóp szenzor\footnote{\href{http://shop.lego.com/en-US/EV3-Gyro-Sensor-45505}{http://shop.lego.com/en-US/EV3-Gyro-Sensor-45505}} és a két nagy szervomotor\footnote{\href{http://lego.wikia.com/wiki/45502\_EV3\_Large\_Servo\_Motor}{http://lego.wikia.com/wiki/45502\_EV3\_Large\_Servo\_Motor}}, amelyek párhuzamosan vannak elhelyezve. Az előbb említett elemekről bővebben szó lesz a \ref{sec:ROBOT:motorok} illetve \ref{sec:ROBOT:szenzorok} fejezetben. Mivel a kerekek párhuzamosan helyezkednek el ezért nem marad egyensúlyi állapotban. E probléma megoldására alkalmas a PID\footnote{\href{https://en.wikipedia.org/wiki/PID\_controller}{https://en.wikipedia.org/wiki/PID\_controller}} szabályzó algoritmus használata, amely ipari körökben elterjedt a viszonylag egyszerű felépítéséért, kezelhetőségéért és implementálhatóságáért. A szabályzó bemeneti értékét az aktuális hiba határozza meg, amelyet négy értékből határozunk meg: szög, szögsebesség, robot sebesség és a robot pozíciója, amelyek külön-külön súlyozva vannak. A szög és szögsebesség érték meghatározásához szükséges a giroszkóp szenzor és a szervomotorok beépített szenzorjai teszik lehetővé a sebesség és pozíció megállapítását.

A robot irányításának érdekében szükséges a PID szabályzó algoritmus módosítása és esetleges újabb szabályzók bevezetése annak érdekében, hogy irányítás alatt ne veszítse el az egyensúlyi állapotát. A felhasználónak lehetőséget ad a projekt Android alkalmazása, hogy hálózaton, socketeken keresztül csatlakozzon a robothoz és az irányításnak megfelelő adatokat továbbítsa. Ezen adatok beviteli módját egy "touch joystick" teszi lehetővé, amellyel négy irányba lehetséges a robot vezérlése. Az adatok védelmét, a tovább bővíthetőséget illetve a szerializációt a Google Protocol Buffers\footnote{\href {https://developers.google.com/protocol-buffers/}{https://developers.google.com/protocol-buffers/}} biztosítja. A Protocol Buffers platform és nyelvfüggetlen, könnyen kezelhető és gyors. Lehetőséget nyújt az adatok tetszőleges felépítésére, amelynek a forráskódját egy speciális generátor segítségével könnyen kigenerálható. E strukturált adatok írását illetve olvasását biztosítja a generált kód.

Az EV3 készlethez biztosított a LEGO MINDSTORMS egy grafikus felületet, amellyel a kisebb korosztály "programozhatja" a saját kezűleg épített robotokat. E projekt esetében leJOS \footnote{\href {https://en.wikipedia.org/wiki/LeJOS}{https://en.wikipedia.org/wiki/LeJOS}} keretrendszer fut az EV3 vezérlőegységen. A leJOS Linux alapú és magába foglalja a JVM-t (Java virtual machine), amely lehetővé teszi, hogy a robot programozható legyen Java-ban. 

A dolgozat 3 fejezetből áll. Az első fejezet bevezetőként szolgál a projektbe.

A második fejezet röviden bemutatja a LEGO megalakulását, a LEGO MINDSTORMS által kifejlesztett generációkat majd bemutatja az EV3 készlethez tartozó motorokat és szenzorokat. Tartalmazza azon eszközök részletes leírását amelyek a projekt által használatosak.

A harmadik fejezet által bemutatásra kerül a PID szabályzó működése és használata e projekt esetén.