\chapter{Következtetések és továbbfejlesztési lehetőségek}
A dolgozat során elkészült egy olyan projekt, amely lehetővé teszi a robot irányítását hálózaton keresztül, Android alkalmazás segítségével. Az alkalmazás lehetőséget ad az automatikus kapcsolat létrehozására a robottal.

Az adatok szerializációját megvalósító Google Protocol Buffers több továbbfejlesztési lehetőséget ad. Az általunk definiált adatstruktúra könnyen bővíthető, így lehetőség van arra, hogy a robot adatokat küldjön az alkalmazásnak a jelenlegi állapotáról, amelyeket az alkalmazás megjelenítene a telefonon . 

Az EV3 ultrahang szenzorja és infravörös szenzorja lehetővé teszi, hogy távolságot mérjünk .Továbbfejlesztési lehetőségként ezek szenzorok egyikét felhasználva lehetséges, hogy a robot irányítás alatt, ha akadályt észlel, akkor megáll, figyelmeztetést küld a felhasználónak és lehetséges irányváltoztatást, annak érdekében, hogy tovább tudjon menni.

Továbbfejlesztés lehetőségként megemlíteném még a fordulás módosítását, egy újabb szabályzó rendszer bevezetésével, hasonlóan, mint az előre-hátra irányításnál, valamint az előre-hátra irányítás finomítását és optimalizálását.



