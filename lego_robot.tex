\chapter{LEGO} \label{ch:ROBOT}

\begin{osszefoglal}
	A fejezet alatt, bemutatásra kerül a LEGO megalakulása, fejlődése. E mellet sor kerül a LEGO MINDSTORMS által kifejlesztett technológiák bemutatása, amelyek közül egy pár felhasználásra kerül a továbbiakban.
\end{osszefoglal}

\section{LEGO megalakulása}\label{sec:ROBOT:lego}
A LEGO története 1932-ben kezdődött,  Ole Kirk Christiansen asztalos vállalkozást alapított Dánia, Billund nevezetű falujában, amely a fa játékok gyártásával foglakozott. 1934-ben vette fel a cég a LEGO nevet, amely a LEG GODT Dán szóból ered. 1947 után kezdtek el műanyagból előállítani játékokat amelyek fő célja az volt, hogy könnyedén egymásba illeszthetőek illetve szétszedhetőek legyenek. Az alapító fia Godtfred Christian lett az igazgató 1958-ban, ekkor alakult meg a ma ismert LEGO vállalat és vezetésével fellendült a cég. Az első számítógép által vezérelt robot 1986-ban jelent meg, amelyet követően 1988-ban a LEGO es az MIT (Massachusetts Institute of Technology) együttműködésével elkezdődött az "inteligens tégla" fejlesztése, mely lehetőséget ad a programozhatóságra.

\section{LEGO MINDSTORMS}\label{sec:ROBOT:mindstorms}
A LEGO első játéka amit forgalmazott Ole Kirk Christians vezetésével a fából készült "LEGO Duck" évekkel később megjelentek a műanyag játékok. Ma már világszerte ismert a LEGO építőjáték, egymással összeilleszthető, kombinálható elemeket tartalmaz így szinte bármi megépíthető belőle és rendkívül hasznos oktatás szempontjából is. 

A LEGO MINDSTORMS, egy programozható robotikai építőkészlet. Lehetőséget ad, hogy megépítsd, programozd és irányítsd a robotot.
1998-ban jelent meg az első generáció az RXC  (Robotic Command eXplorers), akkoriban nagy előrelépés volt mivel képes számítógép nélkül is működni de ma már elavultnak számít.

\subsection{RXC}
Az RXC az elsőgenerácíós LEGO MINDSTORMS. Ma már nem igen használatos, elavult programozható vezérlőegység, mivel csupán 8-bites mikrokontroller és 32KB RAM-al rendelkezik. Tartalmaz három szenzor és három motor csatlakozó port. Ezek mellet egy LCD kijelzőt amin látható volt az akkumulátor töltöttségi szintje, a bemeneti és kimeneti portok állapota illetve egyéb információk megjelenítésere is alkalmas.

\subsection{NXT}
A második generáció, az NXT 2006-ban jelent meg illetve az NXT 2.0 2009-ben amely több építőelemet és szenzort tartalmaz. Több mindenben különbözik az NXT az RXC-hez képest. Szembetűnő esztétikai változások történtek és javítottak a kapacitáson, növelték a portok számát és még más újdonságokat is hozott magával.

Tartalmaz négy szenzor, három motor csatlakozó és egy USB portot. Lehetséges a Bluetooth használata számítógépes csatlakozáshoz, megjelent a grafikus kijelző és tartalmazott egy 8Khz hangszórót. Az NXT 32-bite mikrokontrollerrel, 256KB flash memóriával és 64KB RAM-al rendelkezik.

\subsection{EV3}
A harmadik generáció, az EV3 2013-ban jelent meg, az "EV" evolúciót jelenti és a 3, hogy harmadik generáció. A legfejeltebb programozható építőelem amely kezeli a motorokat, szenzorokat és lehetőséget ad Wi-Fi-n vagy Bluetooth-on keresztül való kommunikációra.

Az EV3 egy ARM9-es nevű processzorral van felszerelve, amely 300Mhz-s, 16MB flash memóriával és 64MB RAM-al rendelkezik. A Linux alapú operációs rendszere nagyban segíti a programozhatóságát. Egy 2.0 USB port van a számítógéppel való kommunikációhoz, aminek átviteli sebessége 480Mbit/s, az SD kártya olvasója 32GB-ot ismer fel. Tartalmaz egy USB portot amely lehetővé teszi, hogy használjunk Wi-Fi dongle-t, ezáltal megkönnyíti a programok kitelepítését, fájlok kezelését és lehetséges a kommunikáció okos készülékekkel is.

Az NXT-hez képest sokat fejlődött, nagyobb a kapacitása, lehetőség van Wi-Fi és SD kártya használatára. Ezek mellet négy szenzor port van három helyett és négy motor csatlakozó port. Nagyobb LCD kijelzőt raktak és a felületén hat gomb van négy helyett, könnyítve az operációs rendszer menüjének kezelését.

\section{Szenzorok}\label{sec:ROBOT:szenzorok}
\section{Motorok}\label{sec:ROBOT:motorok}
