\chapter{Megvalósítás}\label{ch:MEGVALOSITAS}
\begin{osszefoglal}
	A projekt egy EV3 készletből épített kétkerekű egyensúlyozó robot irányítását valósítja meg hálózaton keresztül, telefonos alkalmazáson segítségével. E fejezet alatt bemutatásra kerülnek a megvalósítás során felmerült problémák, ezek megoldása és a felhasznált technológiák.
\end{osszefoglal}
\section{EV3 programozása}\label{sec:MEGVALOSITAS:lejos}
A LEGO MINDSTORMS kifejlesztett egy programozási környezetet, mely célja, hogy a megépített robotot különböző funkcionalitásokkal lehessen felruházni. E környezet lehetővé teszi a kisebb korosztály számára is a robotok programozását. Különböző grafikus elemekből úgynevezett blokkokból épül fel a program, amely USB-n keresztül kitelepíthető az EV3 vezérlőegységen futó LEGO MINDSTORMS által fejlesztett firmware.
Az előbb említett programozási környezet nem alkalmas komplexebb problémák megoldására. Ezért több firmware-t is kifejlesztettek melyek magas szintű programozási nyelveket támogatnak. Esetünkben a leJOS firmware-t használjuk.

A leJOS firmware-t José Solórzano hozta létre 1999 végén és azóta is folyamatosan fejlesztik. Linux alapú, nyílt forráskódú, magába foglalja a JVM-t (Java virtual machine), a neve is rámutat a Java programozhatóságra JOS(Java Operating System). Lehetővé teszi, a robot programozását Java-ban, támogatja a objektum orientált programozást. Mindezek lehetővé teszik a socket alapú komunikációt, szinkronizálhatóságot, szálak alkalmazását, Java típusok használatát és támogatja az EV3 szenzorokat.

Annak érdekében, hogy az EV3 vezérlőegységen futtassuk és könnyedén kitelepitsük a programokat az Eclipse IDE fejlesztői környezetre van szükség és a leJOS plugin-ra.

Mivel az EV3 vezérlőegységen az alapértelmezett firmware van telepítve ezért külön SD kártyára felkel telepíteni a leJOS-t. Legalább 2GB-os SD kártya de ne legyen 32GB-nál nagyobb és ne SDXC típusú legyen, mert nem ismeri fel az EV3 hardware. Az SD kártyát szükséges formázni FAT32 típusú partícióra. A leJOS számítógépre való telepítése során szükség lesz az 1.7 JDK-ra(Java Development Kit). Az előkészített program segítségével feltelepíthető a leJOS firmware az SD kártyára, ehhez még kell a JRE(Java Runtime Environment) is. Sikeres telepítés után az SD kártyát behelyezve az EV3 vezérlőegységbe elindítható a firmware, ha az alapértelmezett rendszer indul el akkor megkel ismételni az SD kártyára való telepítést. Ezt követően telepítsük az Eclipse plugin-t majd berakjuk az EV3\_HOME-t környezeti változónak és a bin könyvtárat a path-be.

\section{Androidos alkalmazás és kommunikáció}\label{sec:MEGVALOSITAS:android}
A projekt része egy telefonos alkalmazás, mely célja hogy hálózaton keresztül kapcsolódjon a robothoz és kommunikáljon vele. Az alkalmazás Android stúdióba készítettük és a kommunikációt java socketen keresztül valósítottuk meg, amit a leJOS, Linux alapú firmware tesz lehetővé.

Az alkalmazást elindítva megadhatjuk a robot IP és PORT címét amin keresztül csatlakozik. A csatlakozás során ellenőrizzük a bekért adatok helyes formátumát és hogy lehetséges vagy sem a kapcsolat. Az alkalmazás kezelését elősegíti egy általunk létrehozott "Remember me" funkcionalitás, mely célja hogy a legutóbbi IP és PORT címet visszatöltse az alkalmazás élindításakor. E megvalósítása a SharedPreferences API-n keresztül történik, érték és kulcs párok alapján tárolódnak fájlba az adatok. Ezen adatok hozzáférési pontja a SharedPreferences objektum, amely könnyen kezelhető metódusokat biztosít ezek olvasására illetve írására.

A sikeres kapcsolódást követően egy 2D-s joystick segítségével lehet irányítani a robotot négy irányba. A joystick vizuális megjelenítésére két kört rajzolunk ki a telefon képernyőjére canvas segítségével, amely felületéhez hozzárendeljük a megfelelő eseményfigyelőt(OnTouchListener). Annak érdekében, hogy a felhasználó ne tudja kimozdítani a nagyobb körön belüli kisebb kört átalakításokat végzünk koordináták között.

A képernyőt megérintve az eseményfigyelő által megkapjuk az $x$ és $y$ koordinátákat, ezeket a pontokat átalakítjuk polárkoordinátákba: $$(x,y) \Longrightarrow (r,\varphi)$$ $$r=\sqrt{x^2+y^2}$$ $$\varphi=arctg(y,x)$$ Tudva a két kör sugarának különbségét és a polárkoordinátákat, leellenőrizhető hogy a kis kör nagy kör sugarán kívül esik vagy sem. Abban az esetben ha a nagy körön kívül esik a kirajzolási pont, akkor a sugár mentén rajzoljuk ki a kisebb kört a szög függvényében. Ehhez szükséges polárkoordinátából átalakítani euklideszi koordinátába:  $$(r,\varphi) \Longrightarrow (x,y)$$ $$x=R*cos(\varphi)$$ $$y=R*sin(\varphi),$$ ahol R a két kör sugarának különbsége.

Tudva a mozgatás irányát, socketen keresztül küldjük a megfelelő adatokat. Az adatok szerializációját illetve titkosítását a Google Protocol Buffers által biztosítjuk, amely lehetővé teszi, hogy megszerkesszük az adatok struktúráját majd egy speciális generátorral létrehozzuk ezen strukturált adatok kezelésére a hozzá tartozó Java osztályt. E létrehozott osztályon keresztül könnyedén kezelhetjük a strukturált adatokat.

A Google Protocol Buffers használatához létre kell hozzunk egy \texttt{.proto} \ref{gpb} kiterjesztésű állományt, amelyben definiáljuk az adataink struktúráját. Az állomány első sorában deklaráljuk a csomag(package) nevét, második sorban konkrétan megadjuk a Java csomag hierarchiáját és a harmadik sorban megadjak az osztály nevet, amellyel rendelkezni fog a legenerált osztály. Abban az esetben ha nem adunk meg konkrét nevet a \texttt{java\_outer\_classname} mezőben akkor a \texttt{.proto} fájl nevét fogja megkapni. A további sorokban megadjuk az adattagokat, amelyek rendelkeznek típus névvel ez lehet bool, int32, float, double és string. Minden attribútumnak megadunk egy számot, amely egyedi azonosítóként szerepel a bináris kódolás során. Ezeken kívül megadható három típus mező minden attribútumnak, amelyek a következők: \texttt{required, optional, repeated}. A \texttt{required} mezővel beállíthatjuk, hogy az adott attribútum kötelezően értéket kapjon különbem RuntimeException vagy IOException hibát eredményez.  A \texttt{optional} mezőt annak az adattagnak állítjuk be, amely nem biztos, hogy értéket kap futási időben, ebben az esetben megadhatunk \texttt{.proto} állományban egy alapértelmezett értéket ennek az adattagnak. A \texttt{repeated} típussal lehetséges annak a jelzése, hogy az adott adattag ismétlődni fog.

\lstinputlisting[label=gpb,caption= Az adatok strukturáját definiáló .proto állomány, language=Java]{progfiles/dataProtos.proto}

\section{Egyensúlyozási algoritmus módosítása}\label{sec:MEGVALOSITAS:pidModositas}





 
